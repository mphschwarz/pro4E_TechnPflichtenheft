\chapter{Auftrag}
% ================ Einstellungen =======================
\thispagestyle{fancy} \setcounter{page}{1} \cfoot{\thepage} \renewcommand{\footrulewidth}{0.4pt} \rhead{\slshape Auftrag}
% ======================================================
Der Besuch eines Kunstmuseums ist weitgehend nicht geführt. Der Besucher bezahlt den Eintritt, ohne zu wissen, was das Museum alles bietet. Für einen Laien resultiert demnach eher ein zielloses Umherschweifen, wodurch sich dessen Fokus weniger auf die bedeutsamen Kunstobjekte richtet. Im Endeffekt bleibt mehr der Besuch in Erinnerung und weniger die Kunstobjekte mit deren Geschichte.
\\[0.5cm]
Der Museumsbesuch kann dem Individuum besser angepasst werden, wie z. B. seinen diskreten Interessen und dessen bevorzugten Sprache. Somit kann nur für die Räume bezahlt werden, welche interessant erscheinen. Anstelle eines gewöhnlichen Tickets soll am Museumseingang ein \textbf{Dojo}\footnote{personalisierbarer Museumsführer} übergeben werden, welches dann die Zutrittsberechtigungen regelt und die Informationen der Kunstobjekte, die sich in der Nähe befinden, über Körperschall\footnote{Übertragung über den Schädelknochen zum Gehör} überträgt. Dadurch wird die Hörbarkeit der Umgebungsgeräusche gewährleistet, anders als bei Kopfhörern. Sollten sich die Ausstellungen des Museums ändern, können einfach neue Daten auf das Dojo geladen werden.
\\[0.5cm]
Bei größerem Interesse des Besuchers an einem Kunstobjekt kann dieser einen \glqq Like-Button\grqq\: drücken und dieses dann quittieren. Beim Zurückgeben des Dojos kann daraus eine persönlich zugeschnittene Museums-History mit den wichtigsten Informationen über die Kunstobjekte zusammengestellt werden.
\\[0.5cm]
Im Verlaufe des Projektes 4 des Studiengangs Elektro- und Informationstechnik geht es um die technische Realisierung des Dojos. Dabei soll das Dojo über einen Akku mit Ladeeinrichtung, Kommunikationsmodule, Bedientasten wie Play/Pause, Lautstärkenregelung und den schon genannten \glqq Like-Button\grqq\: verfügen. Dazu gehören eine Firmware, welche über die Kommunikationsmodule (z. B. Bluetooth) die Zutrittsberechtigungen, wie auch die Ausgabe von Audiofiles (z. B. mp3, wave, ad4) steuert und eine Software, die die allgemeine Verwaltung des Dojos regelt (z. B. Daten Up- bzw. Download von Audiofiles und Likes). 