\chapter{Konzept}
% ================ Einstellungen =======================
\thispagestyle{fancy} \rhead{\slshape Konzept}
% ======================================================
\section{Blockschema}
\begin{figure}[H]
\centering
\includegraphics[width=\textwidth]{Bilder/p4_flowchart.png}
\caption{Blockschema}
\label{fig:flowchart}
\end{figure}
\vspace*{-0.6cm}
\newpage
\paragraph*{Aufbereitung:}
Beim Eintritt in das Museum wird das Dojo individuell angepasst. Dafür werden vom PC aus die Audiofiles mit einer dazugehörigen Korrespondenztabelle\footnote{enthält die Zuordnungen der ID. Nummern der BT-Beacons mit den Audiofiles } auf das Dojo geladen. Dabei werden die Audiofiles direkt auf der $\mu$SD-Karte und die Tabelle auf dem internen EEPROM des Mikrocontrollers gespeichert. Als Ausgangszustand ist der PC mit dem Mikrocontroller verbunden. Per Software kann der USB-Switch geschaltet werden, um die Daten auf die $\mu$SD-Karte zu laden. So kann der Dojo von der PC Software direkt verwaltet und konfiguriert werden. Ansonsten ist die $\mu$SD-Karte mit dem Audioboard verbunden, damit dieses die Audiofiles abspielen kann.
\paragraph*{Benutzung:}
Wenn der Besucher in die Nähe eines Kunstobjektes kommt, soll über das Bluetoothmodul dessen BT-Beacon erfasst werden. In der Firmware des Mikrocontrollers wird über die stärke der Signalleistung des BT-Beacons verifiziert, ob das Objekt nahe genug, oder welches näher ist. Über den Vibrationsmotor und den LEDs wird dann dem Besucher mitgeteilt, dass hier abrufbare Information ist. Gleichermassen würde auch die Zutrittskontrolle erfolgen, indem die empfangenen ID. Nummern der BT-Beacons mit den abgespeicherten abgeglichen werden. Die Objekte können über den Like-Button geliked werden, wodurch die einzigartige ID. Nummer des BT-Beacons im internen EEPROM abgespeichert wird.
\paragraph*{Abgabe:}
Am Schluss können die im int. EEPROM gespeicherten Likes ausgewertet und eine Broschüre mit den Präferenzen des Besuchers zusammengestellt werden.
%%%%%%%%%%%%%%%%%%%%%%%%%%%%%%%%%%%%%%%%%%%%%%%%%%%%%%%%%%%%%%%%%%%%%%%%%%
\section{Hardware}
\subsection{BT-Beacon}
Diese kleinen Geräte können an den Ausstellungsstücken als Signalgeber im Museum angebracht werden. Für dieses Projekt wird ein Minew E7 Bluetooth Beacon verwendet, da dieser Beacon über die zwei bekanntesten Protokolle iBeacon und Eddystone verfügt. In der zugehörigen kostenlosen Konfigurations-App kann eingestellt werden, welches Protokoll gesendet wird. Der Beacon kann auch beide Protokolle gleichzeitig senden, bei Bedarf mit unterschiedlichen Signalstärken und Sendeintervallen. Ausserdem hat dieser Beacon eine Reichweite von 100m, ist wasserfest und verfügt sogar über einen Temperatursensor.
\newpage

\subsection{Bluetoothmodul}
\begin{table}[H]
\footnotesize
\begin{tabular}{|l|lll|}
\toprule
\textbf{Spezifikationen} & \textbf{HM-11 BLE} & \textbf{ISP1507 } & \textbf{SESUB-PAN-T2541} \\ 
\hline 
Version/Klasse & V4.0 BLE  & V5.0 BLE & V4.0 BLE \\ 
\hline 
Reichweite & max. 30m & max 100m & max 10m \\ 
\hline 
Sendeleistung & 23-6 dbm & max 4 dbm & 0 dbm \\ 
\hline 
Unterstützung AT Kommandos & Ja & unbekannt & Ja \\ 
\hline 
Master / Slave & Beide & Beide & Beide \\ 
\hline 
Versorgungsspannung & 3.3 V DC & 1.7 V to 3.6 V DC & 2 V to 3.6 V DC \\ 
\hline 
Kommunikationsprotokoll & UART  & UART, SPI, I2C, PDM & I2C, SPI, UART \\ 
\hline 
Äussere Abmessungen & (13.5 x 18.5 x 2.3) mm & (8 x 8 x 1) mm & (4.6 x 5.6 x 1) mm \\ 
\hline 
Preis & 11.80 CHF & 13.20 CHF & 259.30 CHF \\ 
\bottomrule
\end{tabular} 
\caption{}
\label{tab:bluetoothmodul}
\end{table}
Da Beacons in kurzen, regelmässigen Abständen im 2.4 GHz Band eine Unique ID senden, benötigt der Dojo ein Bluetooth-Modul mit UART-Schnittstelle, um diese Information zu empfangen. 
Für den Prototyp wurden drei Bluetooth-Module tabellarisch verglichen, um eine geeignete Auswahl zu treffen. Auch wenn das HM-11 BLE Bluetooth-Modul die grössten Dimensionen hat, punktet es mit der Reichweite, der Unterstützung der AT Kommandos sowie dem günstigen Preis. 

\subsection{ICSP-Header}
\begin{wrapfigure}{r}{0.3\textwidth}
	\vspace{-40pt}
  	\begin{center}
		\includegraphics[scale=0.6]{Bilder/icsp_header.png}
  	\end{center}
 	\vspace{-20pt}	
	\caption{ICSP-Header}
  	\vspace{-2cm}
	\label{fig:icsp_header_topview}
\end{wrapfigure}
Damit die Firmware auf dem Dojo bearbeitbar bleibt, wird ein ICSP\footnote{In-Circuit-Serial-Programming}-Header verwendet (oder auch ISP\footnote{In-System-Programming}) \cite{ispwiki}. Damit besteht die Möglichkeit, den Mikrocontroller nach der Installation in das komplette System des Dojos programmierbar zu halten. Über eine SPI-Kommunikationsschnittstelle wird der ICSP-Header an den Mikrocontroller angeschlossen. In der Abbildung \ref{fig:icsp_header_topview} sind die Pinanschlüsse eines sechspoligen ICSP-Headers zu sehen.\\
\vspace{1cm}
\subsection{Audioboard}
Das Audioboard beinhaltet den WTV020SD-20S Audiochip. Dieser ist ein kleiner und einfacher IC für die Wiedergabe von Audiofiles. Nach Datenblatt sind mehrere unterschiedliche \textit{Modes} möglich, wobei für das Dojo der \textbf{two line serial mode} verwendet wird. Somit kann der WTV020SD Chip über den Mikrocontroller gesteuert werden und dann Audiofiles, egal welcher Adresse auf der $\mu$SD-Karte abspielen. Der Audiochip ist kompatibel für $\mu$SD-Karten mit Speicherkapazitäten bis zu 1GB. 
\newpage
\subsection{Audiooutput}
Hierfür wird ein Knochenleiter (Bone conductor) verwendet. Dieser übermittelt das Audiosignal über Körperschall des Schädelknochens an das Gehör weiter.
\newline
Der Knochenleiter\footnote{den von den Dozenten zur verfügung gestellten Knochenleiter} wurde an eine Schaltung mit dem Miniverstärker LM386 angeschlossen, um dessen Verbrauch zu messen. Die Lautstärke wurde als Referenz über einen Laptop geregelt, wobei bei einer Lautstärkeneinstellung von 60 Prozent eine maximale Scheinleistung von 0.263VA und einen Effektivstrom von 0.195A erreicht wurde. Bei dieser Lautstärke ist es für eine klare Audio-Übertragung ausreichend. Für den Energieverbrauch wird mit 0.3W und im worst case Szenario mit 0.5W gerechnet.

\subsection{$\mu$SD-Karte}
Als externes Speichermedium wird eine $\mu$SD-Karte mit einer Speicherkapazität von 1GB verwendet.

Die Audiofiles werden bei Ausstellungsstart, also wenn das Museum eine neue Ausstellung eröffnet, auf die Dojos geladen. Unter der Annahme, dass es im Museum 100 Ausstellungsobjekte mit einer Beschreibung von jeweils zwei Minuten gibt, sollen auf der SD-Karte die Dateien für alle Objekte in zwei Sprachen gespeichert werden können. Aufgrund dieser Annahmen, wurde folgende Berechnung durchgeführt:
\subsubsection{Laufzeit}
Die Qualität der Audiofiles soll ungefähr in CD-Qualität erfolgen. Daraus errechnet sich nach \ref{equ:datenrate} eine Datenrate von 705.6kBit/s bei einer Bitauflösung (\textit{resolution}) von 16Bit, einer Samplefrequenz (\textit{samplerate}) von 44.1kHz und einem Kanal (\textit{channels}$\rightarrow$mono).
\begin{equation}
resolution * sample rate * channels = data rate\;[Bit/s]
\label{equ:datenrate}
\end{equation}
Daraus resultiert eine relativ hohe Qualität, welche je nach Aufnahme vom \textbf{Benutzer} angepasst werden kann. Zum Beispiel könnte die \textit{resolution} auf 8Bit, oder auch die samplerate bei der Aufnahme der Audiofiles reduziert werden. Unkomprimiert im \textit{.wav}-Format würde sich mit den 705.6kBit/s bei 1GB Speicherkapazität eine Laufzeit von \textbf{11'337.9 Sekunden} ergeben (ca. 3 Stunden 9 Minuten).
\\[0.5cm]
Nun kann die Bandbreite der Daten noch komprimiert werden. Dies kann anhand der Ansprüche des Benutzers variieren, jedoch leidet dann die Qualität. Um annähernd CD-Qualität beizubehalten, kann im \textit{.mp3}-Format die Datenrate auf 192kBit/s komprimiert werden \cite{koepenick.netkeineAngaben}. Dies entspricht einer Einsparung von 72.8\% und erhöht die Laufzeit auf \textbf{41'666.7 Sekunden} (ca. 11 Stunden 34 Minuten), was für das Szenario ausreichen sollte\footnote{diese Angaben variieren je nachdem wie der Benutzer die Audiofiles abspeichert! Dies sind nur Richtwerte.}.
\subsubsection{Datenübertragungszeit}
Mit der Datenrate des VUB300 (USB zu SDIO Bridge) von 200Mbps, würde es ca. 40 Sekunden dauern, um die 1GB $\mu$SD-Karte vollzuschreiben.
\newpage
\subsection{Tasten}
Für den Nutzer steht auf dem Dojo ein Tastenfeld zur Verfügung:\\
\begin{itemize}
	\item Play/Pause
	\item Volume up/down
	\item Like
	\item Vibro on/off
	\item Power on/off\\
\end{itemize}
\subsection{Energieversorgung}
Das Dojo soll durch einen Li-Ion-Akku mit Energie versorgt werden. Die Akkulaufzeit muss die Dauer mindestens eines Museumrundgangs mit entsprechender Nutzung abdecken, wünschenswert ist eine Akkulaufzeit für einen ganzen Museumstag. Ausserdem wird eine USB-Schnittstelle (Typ C) zur Verfügung stehen, um den Akku nach Rückgabe und Auswertung des Dojos wieder aufzuladen. Die benötigte Kapazität wurde folgendermassen abgeschätzt:
\begin{table}[H]
	\centering
	\begin{adjustbox}{max width=\textwidth}
	\begin{tabular}{l|c|c|c|c}
	\hline
	\textbf{Bezeichnung} & \textbf{Stromaufnahme {[}A{]}} & \textbf{Betriebsdauer {[}\% bei Betrieb{]}} & \textbf{Betriebsmodus {[}\% bei Betrieb{]}} & \textbf{Stromaufnahme geschätzt {[}A{]}} \\
	\hline
	WTV020SD & 3 10\textsuperscript{-6} & 33\% & 100\% & 1 10\textsuperscript{-6} \\
	\hline
	Atmega328 & 2.4 10\textsuperscript{-4} & 100\% & 20\% & 4.8 10\textsuperscript{-5} \\
	\hline
	Vibra-Motor & 6 10\textsuperscript{-2} & 0.1\% & 100\% & 6 10\textsuperscript{-5} \\
	\hline
	Körperschall-Aktor & 2 10\textsuperscript{-1} & 33\% & 50\% & 3.3 10\textsuperscript{-2} \\
	\hline
	HM-11 BLE & 8.5 10\textsuperscript{-3} & 50\% & 100\% & 4.3 10\textsuperscript{-3} \\
	\hline                                                                                               
	\end{tabular}

\end{adjustbox}
\caption{}
\label{tab:energieversorgung}
\end{table}
Addiert ergibt sich ein Strombedarf von ca. 400mAh für einen Betrieb von 10h.
Die Werte beruhen auf Datenblätter-Angaben zum Strombedarf der Bauteile sowie auf Schätzwerten, wie lange das entsprechende Bauteil in Betrieb ist. Der Strombedarf von Bauteilen, die nur in Betrieb sind, wenn das Dojo per USB angeschlossen ist, wurde in der Berechnung nicht berücksichtigt. Als Rahmen wurde angenommen, dass der Museumsbesucher während einem Drittel seiner Aufenthaltszeit im Museum das Dojo in Betrieb hat. Am stärksten ins Gewicht fällt die Versorgung des Körperschallaktors, da dieser eine lange Betriebsdauer kombiniert mit einem relativ hohen Strombedarf hat. 
\subsubsection{Akku}
Im Handel sind Akkus vom Typ LI14500 mit einem Durchmesser von 14 mm und einer Länge von 50 mm erhältlich. Wir haben uns für den «Spezial-Akku 14500 Kabel Li-Ion Emmerich LI14500 3.7 V 800 mAh», erhältlich bei Conrad (1221199-62) zum Preis von CHF 13.95, entschieden. Dieser Akku ist mit einer integrierten Über- und Tiefentladeschutzschaltung ausgerüstet und verfügt über Kabelanschlüsse, die zumindest für den Testaufbau einen Vorteil in der Handhabung bieten. Mit 800 mAh verfügt dieser Akku über eine ausreichend grosse Kapazität um das Dojo während eines ganzen Museumstages ohne Zwischenladung einzusetzen.
\subsubsection{Ladung}
Geladen wird der Dojo über den USB-Anschluss. Der MCP7383 von Microchip  überwacht und steuert die Ladung des Akkus. Er arbeitet nach dem constant-current/constant-voltage Prinzip. Da er wenige zusätzliche Bauteile benötigt, ist er ideal geeignet für portable Geräte, wie der Dojo.
\subsubsection{Spannungsregelung}
Der TC1262 von Microchip regelt die Betriebsspannung von 3.3V. Der LDO liefert bis zu 500mA und ist sehr energiesparend.
\section{Software}
\subsection{Software Mikrocontroller}
Der Mikrocontroller steuert das Audioboard, das BT-Modul, die Status LEDs und den Vibrationsmotor.\\[0.5cm]
Der Prototyp benutzt ein Arduino UNO-Board, wird der Mikrocontroller des Arduino UNO's direkt auf dem Print verbaut.
\\[0.5cm]
Für jeden Besucher wird über eine serielle Schnittstelle ein Profil geladen, das beschreibt, welche Audiodatei zu welchem BT-Beacon gehört. \\
Während dem Besuch steuert der Mikrocontroller, welche Datei abgespielt werden soll, je nach dem, bei welchem Beacon der Besucher steht.\\
Zudem speichert der Mikrocontroller, welche Beacons geliked wurden und nach dem Besuch können die gespeicherten Likes wieder über die serielle Schnittstelle abgerufen werden.\\
\newpage
\subsubsection{Allgemeine Funktionsbeschreibung}
Bluetooth-Funktionen:
\begin{itemize}
	\item getClosestBeacon(): gibt die ID des nächsten Beacons zurück
	\item checkBeacon(id): gibt die Signalstärke des Beacons mit der gefragten ID zurück
\end{itemize}
\vspace*{0.2cm}
Audioboard-Steuerung:
\begin{itemize}
	\item playTrack(tracknumber): spielt die Audiodatei der Nummer tracknumber ab
	\item pausePlayback():
	\item resumePlayback():
	\item cancelPlayback():
\end{itemize}
\vspace*{0.2cm}
Interface-Steuerung:
\begin{itemize}
	\item requestManifest():
	\item deliverLikes():
\end{itemize}

\subsection{Software PC}
Über den PC wird das Dojo mit eine seriellen Schnittstelle verwaltet. Er hat alle Audiodateien, in den möglichen Sprachen und für alle Ausstellungen des Museums und stellt für jeden Besucher eine individuelle Korrespondenztabelle zusammen, basierend auf dessen Ticket und Sprachpräferenzen. Diese wird dann über die PC Software auf das Dojo geladen. Am Ende des Besuchs kann die Software die Likes vom Dojo für eine Zusammenstellung (z. B. Broschüre) mit den Interessen des Besuchers downloaden. Die Software wird voraussichtlich in Python programmiert.\\

