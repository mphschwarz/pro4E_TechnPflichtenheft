\chapter{Testkonzept}
% ================ Einstellungen =======================
\thispagestyle{fancy} \rhead{\slshape Konzept}
% ======================================================
\section{Gesamtsystem}
Der Prototyp wird vom Team getestet, indem ein Museumsbesuch simuliert wird und so jede Funktion gebraucht wird. Es wird überprüft, ob die gewünschten Audiodateien abgespielt werden können. Zudem dürfen die nicht gewählten Dateien nicht abgespielt werden. Die Beacon-Erkennung wird simuliert indem drei Beacons in einem Abstand von ca. einem Meter platziert werden. Durch die Erkennung eines Beacons soll ein Vibrationsalarm ausgelöst werden. Der Beacon welcher dem Empfänger am nächsten ist soll gewählt werden und die dazu hinterlegte Audio Datei muss über den Bone Conductor abgespielt werden. Danach werden allfällige Fehler behoben und der Auftraggeber kann das Gerät testen und entscheiden, ob es den Anforderungen genügt.
\section{Hardware}
Während der ganzen Entwicklungsphase werden die einzelnen Komponenten (Bluetooth, USB/UART, Audioboard, etc.) getestet. Auf dem Print werden sie nacheinander in Betrieb genommen, um zu testen, ob sie auch zusammen funktionieren. Die Akku-Ladeschaltung wird zuerst auf einem Entwicklungsboard aufgebaut und mit einem Widerstand eine Last simuliert. So werden mehrere Lade- und Entladezyklen simuliert bevor die Schaltung im Gerät verbaut wird.
\section{Software}
Mit dem Mikrocontroller werden die sogenannten «Likes» simuliert und an den Computer gesendet. So kann getestet werden, ob die Software die «Likes» empfängt und richtig verarbeitet.